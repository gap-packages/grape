%%%%%%%%%%%%%%%%%%%%%%%%%%%%%%%%%%%%%%%%%%%%%%%%%%%%%%%%%%%%%%%%%%%%%%%%%%%%
%
%A  auxil.tex           GRAPE documentation             Leonard Soicher
%
%
%
\def\GRAPE{\sf GRAPE}
\def\nauty{\it nauty}
\def\G{\Gamma}
\def\Aut{{\rm Aut}\,}
\def\x{\times}
\Chapter{Auxiliary functions}

This chapter documents some auxiliary functions used in {\GRAPE},
which may be of wider interest. 

%%%%%%%%%%%%%%%%%%%%%%%%%%%%%%%%%%%%%%%%%%%%%%%%%%%%%%%%%%%%%%%%%%%%%%%%
\Section{Steve Linton's Function SmallestImageSet}

\>SmallestImageSet( <G>, <S> )
\>SmallestImageSet( <G>, <S>, <H> )

Let <G> be a permutation group on $\{1,\ldots,n\}$, and let <S>
be a subset of $\{1,\ldots,n\}$. Then this function returns the
lexicographically least set in the <G>-orbit of <S>, with respect to the
action `OnSets', without explicitly computing this (possibly huge) orbit.

Thus, if <C> is a list of subsets of $\{1,\ldots,n\}$ and we
want to determine a set of (canonical) representatives for the
distinct <G>-orbits of the elements of <C>, we can do this as
`Set(<C>,c->SmallestImageSet(<G>,c))'.

If the setwise stabilizer in <G> of <S> is known, then this should be
given as the optional third parameter, to avoid the recomputation of
this stabilizer.

The function `SmallestImageSet' was written by Steve Linton, based
on his algorithm described in \cite{Lin04}. 

\beginexample
gap> J:=JohnsonGraph(12,5);;
gap> OrderGraph(J);
792
gap> G:=J.group;;
gap> Size(G);
479001600
gap> S:=[67,93,100,204,677,750];;
gap> SmallestImageSet(G,S);
[ 1, 2, 22, 212, 242, 446 ]
\endexample

%%%%%%%%%%%%%%%%%%%%%%%%%%%%%%%%%%%%%%%%%%%%%%%%%%%%%%%%%%%%%%%%%%%%%%%%
\Section{Exact Set-cover}

\>GRAPE_ExactSetCover( <G>, <blocks>, <n> )
\>GRAPE_ExactSetCover( <G>, <blocks>, <n>, <H> )

Suppose <n> is a non-negative integer, <G> is a permutation group
on $\{1,\ldots,n\}$, <blocks> is a list of non-empty subsets
of $\{1,\ldots,n\}$, and the optional parameter <H> (default:
`Group(())') is a subgroup of <G>. 

Then this function returns an <H>-invariant exact 
set-cover of $\{1,\ldots,n\}$, consisting of elements from the union of
`Orbits(<G>,<blocks>,OnSets)', if such a cover exists,
and returns  `fail'  otherwise. An exact set-cover is given as a set of
sets forming a partition of $\{1,\ldots,n\}$.

\beginexample
gap> G:=PSL(2,5);;
gap> GRAPE_ExactSetCover(G,[[1,2,3]],6);
fail
gap> G:=PGL(2,5);;
gap> GRAPE_ExactSetCover(G,[[1,2,3]],6);
[ [ 1, 2, 3 ], [ 4, 5, 6 ] ]
gap> n:=280;;
gap> G:=OnePrimitiveGroup(NrMovedPoints,n,Size,604800*2);
J_2.2
gap> gamma:=First(GeneralizedOrbitalGraphs(G),x->VertexDegrees(x)=[135]);;
gap> omega:=CliqueNumber(gamma);
28
gap> blocks:=CompleteSubgraphsOfGivenSize(ComplementGraph(gamma),n/omega,2);;
gap> Collected(List(blocks,Length));
[ [ 10, 2 ] ]
gap> H:=SylowSubgroup(G,7);;
gap> partition:=GRAPE_ExactSetCover(G,blocks,n,H);;
gap> Collected(List(partition,Length));
[ [ 10, 28 ] ]
gap> Union(partition)=[1..n];
true
\endexample

