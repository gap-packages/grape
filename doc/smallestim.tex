%%%%%%%%%%%%%%%%%%%%%%%%%%%%%%%%%%%%%%%%%%%%%%%%%%%%%%%%%%%%%%%%%%%%%%%%%%%%
%
%A  smallestim.tex           GRAPE documentation             Leonard Soicher
%
%
%
\def\GRAPE{\sf GRAPE}
\def\nauty{\it nauty}
\def\G{\Gamma}
\def\Aut{{\rm Aut}\,}
\def\x{\times}
\Chapter{Steve Linton's function SmallestImageSet}

This chapter documents the straightforward application of Steve Linton's
function `SmallestImageSet' \cite{Lin04}, which is included and used
in {\GRAPE}. The function is of use when classifying objects up to
the action of a given permutation group $G$, when the objects can be
represented as subsets of the permutation domain of $G$.

%%%%%%%%%%%%%%%%%%%%%%%%%%%%%%%%%%%%%%%%%%%%%%%%%%%%%%%%%%%%%%%%%%%%%%%%
\Section{SmallestImageSet}

\>SmallestImageSet( <G>, <S> )
\>SmallestImageSet( <G>, <S>, <H> )

Let <G> be a permutation group on $\{1,\ldots,n\}$, and let <S>
be a subset of $\{1,\ldots,n\}$. Then this function returns the
lexicographically least set in the <G>-orbit of <S>, with respect to the
action `OnSets', without explicitly computing this (possibly huge) orbit.

Thus, if <C> is a list of subsets of $\{1,\ldots,n\}$ and we
want to determine a set of (canonical) representatives for the
distinct <G>-orbits of the elements of <C>, we can do this as
`Set(<C>,c->SmallestImageSet(<G>,c))'.

If the setwise stabiliser in <G> of <S> is known, then this should be
given as the optional third parameter.

\beginexample
gap> J:=JohnsonGraph(12,5);;
gap> OrderGraph(J);
792
gap> G:=J.group;;
gap> Size(G);
479001600
gap> S:=[67,93,100,204,677];;
gap> SmallestImageSet(G,S);
[ 1, 2, 22, 220, 453 ]
\endexample

